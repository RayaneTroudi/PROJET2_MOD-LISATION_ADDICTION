%%%%%%%%%%%%%%%%%%%%%%%%%%%%%%%%%%%%%%%%%%%%%%%%%%%%%%%%%%%%%%%
%
% Welcome to Overleaf --- just edit your LaTeX on the left,
% and we'll compile it for you on the right. If you open the
% 'Share' menu, you can invite other users to edit at the same
% time. See www.overleaf.com/learn for more info. Enjoy!
%
%%%%%%%%%%%%%%%%%%%%%%%%%%%%%%%%%%%%%%%%%%%%%%%%%%%%%%%%%%%%%%%
\documentclass{beamer}
\usepackage[utf8]{inputenc}

\usetheme{Madrid}
\usecolortheme{default}
%Information to be included in the title page:
\title{Modélisation de l'addiction}
\author{Gerbaud Florent \\ Troudi Rayane \\ Zouga Jassim}
\institute{Polyteh Nice}
\date{\today}

%------------------------------------------------------------
%The next block of commands puts the table of contents at the 
%beginning of each section and highlights the current section:

\AtBeginSection[]
{
	\begin{frame}
		\frametitle{Table of Contents}
		\tableofcontents[currentsection]
	\end{frame}
}

\begin{document}
	
	\frame{\titlepage}
	
	%---------------------------------------------------------
	%This block of code is for the table of contents after
	%the title page
	\begin{frame}
		\frametitle{Table of Contents}
		\tableofcontents
	\end{frame}
	%---------------------------------------------------------
	
	
	\section{First section}
	
	\begin{frame}
		\frametitle{Introduction}
		
		In this slide, some important text will be
		\alert{highlighted} because it's important.
		Please, don't abuse it.
		
		\begin{block}{Définition}
			Sample text
		\end{block}
		
		\begin{alertblock}{Important theorem}
			Sample text in red box
		\end{alertblock}
		
		\begin{examples}
			Sample text in green box. The title of the block is ``Examples".
		\end{examples}
	\end{frame}
	
	
\begin{frame}
	\frametitle{Le modèle mathématiques}
	\begin{block}{Équation théorique du modèle}
		\begin{align*}
			&C(t) :=\text{"Intensité de fringale ou de désir"} \\
			&S(t):=\text{"Intensité de Self Contrôle"} \\
			&A(t):=\text{"Passage à l'acte"}\\
			&V(t):=\text{"Etat addictifs"} \\
			&E(t):=\text{"Influences extérieurs"} 
		\end{align*}
	\end{block}
	\begin{block}{Condition initiale}
		\begin{align*}
			&E(0)=E_0, \ C(0)=C_0, \ S(0)=S_0
		\end{align*}
	\end{block}
\end{frame}

\begin{frame}
	\begin{block}{Constante du modèle}
		\begin{align*}
			&S_m\text{:="Self Contrôle max"} \\
			&k\text{:=coefficient du passage à l'acte"} \\
			&h\text{:="Compétition entre S et C"} \\
			&p\text{:="Résilience psychologique"} \\
			&\alpha\text{:="Effet de l'addiction"} \\
			&\gamma \text{:="Accentuation du Désir dû au passage à l'acte"} \\
			&b\text{:="Influence de passée à l'acte"} \\
			&q\text{:="quantité maximum d'ingestion"}
		\end{align*}
	\end{block}
\end{frame}

	
\begin{frame}
	\begin{block}{Définition mathématique du modèle}
		\begin{itemize}
			\item<1-> $C(t+1) = C(t)- \alpha C(t) + \gamma A(t)$
			\item<2-> $S(t+1)=S(t)+p.max\{0,S_{max}-S(t)\}-h.C(t)-k.A(t)$
			\item<3-> $A(t)=A(V)=qV$
			\item<4->$V= max\{1,min\left\lbrace \phi(t),0 \right\rbrace \}$
			\item<5->$E(t)=$
		\end{itemize}
	\end{block}
\end{frame}

\section{Formule}

\begin{frame}
	\frametitle{F}
	
	In this slide, some important text will be
	\alert{highlighted} because it's important.
	Please, don't abuse it.
	
	\begin{block}{Définition}
		Sample text
	\end{block}
	
	\begin{alertblock}{Important theorem}
		Sample text in red box
	\end{alertblock}
	
	\begin{examples}
		Sample text in green box. The title of the block is ``Examples".
	\end{examples}
\end{frame}

\section{Test}

\begin{frame}
	\frametitle{T}
	
	In this slide, some important text will be
	\alert{highlighted} because it's important.
	Please, don't abuse it.
	
	\begin{block}{Définition}
		Sample text
	\end{block}
	
	\begin{alertblock}{Important theorem}
		Sample text in red box
	\end{alertblock}
	
	\begin{examples}
		Sample text in green box. The title of the block is ``Examples".
	\end{examples}
\end{frame}

\end{document}


%%%%%%%%%%%%%%%%%%%%%%%%%%%%%%%%%%%%%%%%%%%%%%%%%%%%%%%%%%%%%%%
%
% Welcome to Overleaf --- just edit your LaTeX on the left,
% and we'll compile it for you on the right. If you open the
% 'Share' menu, you can invite other users to edit at the same
% time. See www.overleaf.com/learn for more info. Enjoy!
%
%%%%%%%%%%%%%%%%%%%%%%%%%%%%%%%%%%%%%%%%%%%%%%%%%%%%%%%%%%%%%%%


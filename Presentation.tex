\documentclass{beamer}
\usepackage[utf8]{inputenc}
\usepackage{etoolbox}
\usepackage{caption}
\usetheme{Madrid}
\usecolortheme{default}
%Information to be included in the title page:
\title{Modélisation de l'addiction}
\author{Gerbaud Florent\\ Troudi Rayane \\Zouga Jassim}
\institute{Polyteh Nice}
\date{\today}

%------------------------------------------------------------
%The next block of commands puts the table of contents at the 
%beginning of each section and highlights the current section:

\AtBeginSection[]
{
	\begin{frame}
		\frametitle{Table of Contents}
		\tableofcontents[currentsection]
	\end{frame}
}

\begin{document}
	
	\frame{\titlepage}
	
	%---------------------------------------------------------
	%This block of code is for the table of contents after
	%the title page
	\begin{frame}
		\frametitle{Table of Contents}
		\tableofcontents
	\end{frame}
	%---------------------------------------------------------
	
	
	\section{Introduction}
	
	\begin{frame}
		\frametitle{Introduction}
		
		\begin{block}{Définition de l'addiction}
			Sample text
		\end{block}
	\end{frame}
	
	
\section{Modèle mathématique}

\begin{frame}
	\frametitle{Le modèle mathématiques}
	\begin{block}{Équation théorique du modèle}
		\begin{align*}
			&C(t) :=\text{"Intensité de fringale ou de désir"} \\
			&S(t):=\text{"Intensité de Self Contrôle"} \\
			&A(t):=\text{"Passage à l'acte"}\\
			&V(t):=\text{"Etat addictifs"} \\
			&E(t):=\text{"Influences extérieurs"}  \\
			&\psi(t):=\text{"État psychologique"}
		\end{align*}
	\end{block}
	\begin{block}{Condition initiale}
		\begin{align*}
			&E(0)=E_0, \ C(0)=C_0, \ S(0)=S_0
		\end{align*}
	\end{block}
\end{frame}

\begin{frame}
	\begin{block}{Constante théorique du modèle}
		\begin{align*}
			&S_m\text{:="Self Contrôle max"} \\
			&k\text{:="coefficient du passage à l'acte"} \\
			&h\text{:="Compétition entre S et C"} \\
			&p\text{:="Résilience psychologique"} \\
			&\alpha\text{:="Effet d'oublie"} \\
			&\gamma \text{:="Accentuation du Désir dû au passage à l'acte"} \\
			&b\text{:="Influence de passée à l'acte"} \\
			&q\text{:="quantité maximum d'ingestion"} \\
		\end{align*}
	\end{block}
\end{frame}


\begin{frame}
	\begin{block}{Définition mathématique du modèle initiale}
		\begin{itemize}
			\item<1-> $C(t+1) = C(t)- \alpha C(t) + \gamma A(t)$
			\item<2-> $S(t+1)=S(t)+p.max\{0,S_{max}-S(t)\}-h.C(t)-k.A(t)$
			\item<3-> $A(t)=A(V)=qV$
			\item<4->$V= min\{1,max\left\lbrace \phi(t),0 \right\rbrace \}$
			\item<5->$E(t)=E_0$
			\item<6->$\psi(t)=C(t)-S(t)-E(t)$
		\end{itemize}
	\end{block}
\end{frame}


\section{Cas sans exposition sociale }

\begin{frame}
	\frametitle{Les paramètres}
\end{frame}

\begin{frame}
	\frametitle{Comment évolue le passage à l'acte en fonction de la fringale ?}
	\begin{minipage}{0.45\linewidth}
		\centering
		\begin{figure}
			\includegraphics[width=\linewidth]{SansExpositionSociale1.png}
			\captionsetup{justification=centering, skip=5pt}
			\caption{Fringale VS passage à l'acte (1)}
		\end{figure}
	\end{minipage}\hfill
	\begin{minipage}{0.45\linewidth}
		\centering
		\begin{figure}
			\includegraphics[width=\linewidth]{SansExpositionSociale2.png}
			\captionsetup{justification=centering, skip=5pt}
			\caption{Fringale VS passage à l'acte (2)}
		\end{figure}
	\end{minipage}
\end{frame}

\begin{frame}
	\frametitle{Comment évolue la vulnérabilité en fonction du self contrôle ?}
	\begin{minipage}{0.45\linewidth}
		\centering
		\begin{figure}
			\includegraphics[width=\linewidth]{SansExpositionSociale1_1.png}
			\captionsetup{justification=centering, skip=5pt}
			\caption{Self contrôle VS Vulnérabilité (1)}

		\end{figure}
	\end{minipage}\hfill
	\begin{minipage}{0.45\linewidth}
		\centering
		\begin{figure}
			\includegraphics[width=\linewidth]{SansExpositionSociale2_2.png}
			\captionsetup{justification=centering, skip=5pt}
			\caption{Self contrôle VS Vulnérabilité (2)}
		\end{figure}
	\end{minipage}
\end{frame}

\section{Cas avec exposition sociale :}

\begin{frame}
	\frametitle{Équation avec l'influence sociale}
	%\framesubtitle{subtitle}
	\begin{block}{Nouvelles équations}
		\begin{enumerate}
			\item $C(t+1) = C(t)- \alpha C(t) + \gamma A(t)$
			\item $S(t+1)=S(t)+p.max\{0,S_{max}-S(t)\}-h.C(t)-k.A(t)$
			\item $\alert{A(t)=A(V)=qV+\frac{R(\lambda(t))}{R_m}q\left( 1-V(t)\right) }$
			\item $V= max\{1,min\left\lbrace \phi(t),0 \right\rbrace \}$
			\item $\alert{E(t+1)=E(t)-m_E}$
			\item $\psi(t)=C(t)-S(t)-E(t)$
		\end{enumerate}
	\end{block}
	\begin{block}{Constante}
		\begin{enumerate}
			\item $m_E:=\text{"évolution de l'influence sociale"}$ 
			\item $R_m:=\text{"Rencontre maximale"}$
		\end{enumerate}
	\end{block}
\end{frame}

\begin{frame}
	\frametitle{Comment évolue le passage à l'acte en fonction de la fringale ?}
	\begin{minipage}{0.45\linewidth}
		\centering
		\begin{figure}
			\includegraphics[width=\linewidth]{AvecExpositionSociale1.png}
			\captionsetup{justification=centering, skip=5pt}
			\caption{Fringale VS vulnérabilité (1)}
		\end{figure}
	\end{minipage}\hfill
	\begin{minipage}{0.45\linewidth}
		\centering
		\begin{figure}
			\includegraphics[width=\linewidth]{AvecExpositionSociale2.png}
			\captionsetup{justification=centering, skip=5pt}
			\caption{Fringale VS vulnérabilité (2)}
		\end{figure}
	\end{minipage}
\end{frame}

\begin{frame}
	\frametitle{Comment évolue la vulnérabilité en fonction du self contrôle ?}
	\begin{minipage}{0.45\linewidth}
		\centering
		\begin{figure}
			\includegraphics[width=\linewidth]{AvecExpositionSociale1_1.png}
			\captionsetup{justification=raggedleft, skip=5pt}
			\caption{Param1}
		\end{figure}
	\end{minipage}\hfill
	\begin{minipage}{0.45\linewidth}
		\centering
		\begin{figure}
			\includegraphics[width=\linewidth]{AvecExpositionSociale2_2.png}
			\captionsetup{justification=centering, skip=5pt}
			\caption{Évolution avec plus d'addiction au départ}
		\end{figure}
	\end{minipage}
\end{frame}

\begin{frame}
	\frametitle{L'influence sociale et les occasions sociales}
	\begin{minipage}{0.45\linewidth}
		\centering
		\begin{figure}
			\vspace{-0.45cm}
			\includegraphics[width=\linewidth]{AvecExpositionSociale1_1_1.png}
			\captionsetup{justification=centering, skip=5pt}
			\caption{Param1}
		\end{figure}
	\end{minipage}\hfill
	\begin{minipage}{0.45\linewidth}
		\centering
		\begin{figure}
			\includegraphics[width=\linewidth]{AvecExpositionSociale2_2_2.png}
			\captionsetup{justification=centering, skip=5pt}
			\caption{Évolution avec plus d'addiction au départ}
		\end{figure}
	\end{minipage}
\end{frame}

\section{Approches thérapeutiques}

\begin{frame}
	\frametitle{Équation avec l'influence sociale de manière périodique}
	%\framesubtitle{subtitle}
	\begin{block}{Les équations qui incluent un traitement psychologique}
		\begin{enumerate}
			\item $C(t+1) = C(t)- \alpha C(t) + \gamma A(t)$
			\item $S(t+1)=S(t)+p.max\{0,S_{max}-S(t)\}-h.C(t)-k.A(t)$
			\item $A(t)=A(V)=qV+\frac{R(\lambda(t))}{R_m}q\left( 1-V(t)\right) $
			\item $V= max\{1,min\left\lbrace \phi(t),0 \right\rbrace \}$
			\item \alert{$E(t+1)=\begin{cases}
					1 \text{ si la semaine est multiple de 5} \\
					E(t)-m_E \ sinon
				\end{cases}$}
			\item $\psi(t)=C(t)-S(t)-E(t)$
		\end{enumerate}
	\end{block}
	\begin{block}{Constante}
		\begin{enumerate}
			\item $m_E:=\text{"évolution de l'influence sociale"}$ 
			\item $R_m:=\text{"Rencontre maximale"}$
		\end{enumerate}
	\end{block}
\end{frame}


\section{Conclusion}

\begin{frame}
	\frametitle{Conclusion}
\end{frame}

\end{document}


